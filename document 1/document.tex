\documentclass{article}
\usepackage[utf8]{inputenc}
\usepackage[T1]{fontenc}
\usepackage[polish]{babel}
\usepackage[margin=1in]{geometry}

\title{Chainship}
\date{}

\begin{document}

\maketitle

\section{Wprowadzenie}

Projekt zakłada stworzenie aplikacji opartej na technologii blockchain, która umożliwi grę w statki pomiędzy dwoma graczami. Gra odbywać się będzie w pełni w przeglądarce, przy użyciu interface'u napisanego w JavaScriptcie i współpracy z portfelem Metamask. Aplikacja będzie wystawiona publicznie jako strona internetowa, jednak każdy użytkownik będzie mógł uruchomić ją lokalnie z pominięciem centralnego serwera, co zminimalizuje konieczność większego zaufania. Wszystkie działania związane z grą, takie jak utworzenie pokoju, przechowywanie danych o ustawieniu statków, strzały, wyniki gry, będą realizowane na kontraktach blockchainowych na Ethereum. Dzięki tej technologii możliwe będzie zapewnienie przejrzystości, bezpieczeństwa oraz decentralizacji całego procesu.

\section{Opis gry}

Gra toczyć się będzie o pulę pieniężną, którą obaj gracze wpłacają przed rozpoczęciem rozgrywki. Zwycięzca otrzymuje całą pulę, z wyjątkiem opcjonalnej prowizji, którą może pobierać właściciel kontraktów.

Gra będzie odbywać się między dwoma graczami. Kluczowe elementy rozgrywki to:
\begin{itemize}
    \item Tworzenie pokoju gry: Jeden z graczy może stworzyć prywatny pokój, zapraszając drugiego gracza za pomocą linku. (opcjonalne) Inną opcją jest połączenie z losowym graczem na podstawie punktów rankingowych.
    \item Faza ustawienia statków: Obaj gracze wybierają pola na planszy, na których umieszczą swoje statki. Ten etap będzie odbywać się w aplikacji w interfejsie graficznym.
    \item Faza strzałów: Po ustawieniu statków gracze na przemian strzelają do pól na planszy przeciwnika. Wyniki strzałów będą natychmiastowo rejestrowane na blockchainie, a aplikacja przeglądarkowa będzie dostarczać użytkownikom aktualne dane o stanie gry.
    \item Dowodzenie wygranej: Po zakończeniu gry, zwycięzca musi udowodnić, że poprawnie odpowiadał na pytania przeciwnika. Dopiero w tym momencie każdy z graczy zdradza jak wygląda jego plansza, co sprawia, że żaden z graczy nie może oszukiwać.
    \item Ranking graczy: Aplikacja będzie prowadziła ranking najlepszych graczy, bazując na liczbie wygranych gier oraz wysokości ich wygranych. Ranking będzie widoczny dla wszystkich graczy i stanowił element rywalizacji.
\end{itemize}

\section{Korzyści z realizacji projektu na blockchainie}

Realizacja tej gry na blockchainie przynosi szereg korzyści, które wyróżniają ją na tle tradycyjnych gier internetowych. Oto kluczowe argumenty, dlaczego blockchain jest idealnym rozwiązaniem w tym przypadku:

\subsection{Decentralizacja i bezpieczeństwo}

Blockchain zapewnia, że gra jest całkowicie zdecentralizowana, eliminując potrzebę istnienia centralnego serwera. Gracze nie muszą ufać żadnej pojedynczej stronie czy organizacji, ponieważ wszystkie dane związane z grą (takie jak ustawienie statków, strzały, wynik gry) są przechowywane w rozproszony sposób na blockchainie. Każdy ruch gracza jest rejestrowany i weryfikowany przez cały system, co zapewnia transparentność i bezpieczeństwo. Ponad to, brak serwera oznacza również, że nikt poza graczem nie zna ułożenia statków na jego planszy przez zakończeniem fazy strzałów, dzięki czemu współpraca z twórcą aplikacji nie mogłaby przynieść graczowi żadnej korzyści. Brak centralnego serwera oznacza również, że nie ma jednego punktu awarii. Gra jest całkowicie rozproszona, co oznacza, że nawet w przypadku problemów z infrastrukturą, gra nadal może funkcjonować, a dane pozostaną nienaruszone.

\subsection{Transparentność wyników}

Dzięki zastosowaniu blockchaina wszystkie wyniki gier są zapisane w sposób jawny, a każda zmiana stanu gry (np. trafienie w statek) jest publicznie dostępna. Zwycięzca gry nie tylko otrzymuje pulę, ale ma również pełną możliwość udowodnienia swojej wygranej poprzez weryfikację zapisanych na blockchainie transakcji. To eliminuje ryzyko oszustwa czy manipulacji. Ponad to, system rankingowy jest w pełni jawny i deterministyczny, więc gracze mają pewność, że pozycje użytkowników w rankingu są uczciwie zdobyte.

\subsection{Prowizje i monetyzacja}

Zastosowanie blockchaina umożliwia właścicielowi kontraktów pobieranie prowizji od wygranych. System ten jest bezpieczny i automatyczny, co eliminuje potrzebę jakiejkolwiek zewnętrznej kontroli nad procesem wypłat. Prowizja będzie ustalana w sposób przejrzysty, a użytkownicy mogą mieć pewność, że zasady są jasno określone i egzekwowane.

\section{Zalety projektu}

Projekt posiada wiele mocnych stron, które przyciągają uwagę zarówno graczy, jak i inwestorów. Oto najważniejsze z nich:

\subsection{Innowacyjność}

Połączenie klasycznej gry w statki z technologią blockchain sprawia, że projekt wyróżnia się na tle tradycyjnych gier online. Gracze mogą cieszyć się rozrywką, mając jednocześnie pewność, że ich dane są przechowywane w sposób bezpieczny, transparentny i dostępny.

\subsection{Rywalizacja}

Ranking graczy i rywalizacja o pulę pieniężną tworzy elementy gier hazardowych, ale w bezpiecznej, transparentnej i zdecentralizowanej formie. To z pewnością przyciągnie osoby szukające nowych wyzwań oraz tych, którzy lubią rywalizować w grach online.

\subsection{Potencjał na rynku blockchain}
Z uwagi na rosnące zainteresowanie grami na blockchainie i decentralizacją, projekt ma duży potencjał na rynku. Gry oparte na blockchainie są coraz bardziej popularne, a ten projekt może przyciągnąć zarówno entuzjastów blockchaina, jak i miłośników klasycznych gier komputerowych.

\subsection{Zautomatyzowana monetyzacja}

Możliwość pobierania prowizji od wygranych w połączeniu z automatycznym procesem wypłat stanowi solidny fundament monetyzacji. Gra będzie mogła generować stałe przychody dla właściciela kontraktów, co zapewnia długoterminową opłacalność projektu.

\subsection{Możliwość rozwoju}

Projekt może być rozwijany o dodatkowe funkcjonalności, takie jak nowe rodzaje gier, bazując na istniejącym rankingu graczy. Możliwość dodania nowych opcji rozgrywki sprawia, że projekt ma ogromny potencjał do rozwoju i dostosowania do potrzeb użytkowników.

\section{Podsumowanie}

Realizacja gry w statki na blockchainie ma ogromny potencjał, łącząc klasyczną rozrywkę z nowoczesną technologią. Dzięki zastosowaniu technologii blockchain możliwe będzie zapewnienie bezpieczeństwa, transparentności oraz decentralizacji, co stanowi istotną przewagę nad tradycyjnymi grami online.

\end{document}
