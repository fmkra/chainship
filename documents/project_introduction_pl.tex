\documentclass{article}
\usepackage[utf8]{inputenc}
\usepackage[T1]{fontenc}
\usepackage[polish]{babel}
\usepackage[margin=1in]{geometry}

\title{Chainship}
\date{}

\begin{document}

\maketitle

\section{Wprowadzenie}

Projekt zakłada stworzenie aplikacji umożliwiającej grę w statki, która wykorzystuje technologię blockchain do stworzenia bezpiecznego, transparentnego i wciągającego doświadczenia dla graczy. Wykorzystując JavaScript i integrację z portfelem Metamask, Chainship oferuje grę dostępną bezpośrednio w przeglądarce, z możliwością uruchomienia lokalnego, eliminując potrzebę zaufania scentralizowanemu serwerowi. Wszystkie aspekty gry – od tworzenia pokoju, przez ustawianie statków, po rejestrowanie strzałów i wyników – są zabezpieczone przez smart kontrakty działające na Ethereum. Chainship to nie tylko gra; to demonstracja przyszłości gamingu, gdzie uczciwość, decentralizacja i bezpieczeństwo są priorytetem.

\section{Opis gry}

Gracze będą rywalizować o pulę pieniężną, wpłacaną przed rozpoczęciem rozgrywki, a zwycięzca zgarnia całą pulę (pomniejszoną o ewentualną prowizję dla właściciela kontraktów).

Gra będzie odbywać się między dwoma graczami. Kluczowe elementy rozgrywki to:
\begin{itemize}
    \item \textbf{Tworzenie pokoju gry:} Gracze mogą tworzyć prywatne pokoje i zapraszać znajomych lub dołączać do losowych przeciwników na podstawie rankingu.
    \item \textbf{Faza ustawienia statków:} Intuicyjny interfejs graficzny umożliwia graczom strategiczne rozmieszczanie statków na planszy.
    \item \textbf{Faza Strzałów:} Gracze na przemian oddają strzały, a wyniki są natychmiast rejestrowane na blockchainie, zapewniając transparentność i eliminując oszustwa.
    \item \textbf{Dowodzenie Wygranej:} Zwycięzca musi udowodnić poprawność swoich odpowiedzi, ujawniając swoją planszę dopiero po zakończeniu gry, co gwarantuje uczciwość rozgrywki.
    \item \textbf{Ranking Graczy:} System rankingowy, oparty na wygranych grach i zdobytych nagrodach, dodaje element rywalizacji i prestiżu.
\end{itemize}

\section{Korzyści z realizacji projektu na blockchainie}

Realizacja tej gry na blockchainie przynosi szereg korzyści, które wyróżniają ją na tle tradycyjnych gier internetowych. Oto kluczowe argumenty, dlaczego blockchain jest idealnym rozwiązaniem w tym przypadku:

\subsection{Decentralizacja i bezpieczeństwo}

Blockchain zapewnia, że gra jest całkowicie zdecentralizowana, eliminując potrzebę istnienia centralnego serwera. Gracze nie muszą ufać żadnej pojedynczej stronie czy organizacji, ponieważ wszystkie dane związane z grą (takie jak ustawienie statków, strzały, wynik gry) są przechowywane w rozproszony sposób na blockchainie. Każdy ruch gracza jest rejestrowany i weryfikowany przez cały system, co zapewnia transparentność i bezpieczeństwo. Ponad to, brak serwera oznacza również, że nikt poza graczem nie zna ułożenia statków na jego planszy przez zakończeniem fazy strzałów, dzięki czemu współpraca z twórcą aplikacji nie mogłaby przynieść graczowi żadnej korzyści. Brak centralnego serwera oznacza również, że nie ma jednego punktu awarii. Gra jest całkowicie rozproszona, co oznacza, że nawet w przypadku problemów z infrastrukturą, gra nadal może funkcjonować, a dane pozostaną nienaruszone.

\subsection{Transparentność wyników}

Dzięki zastosowaniu blockchaina wszystkie wyniki gier są zapisane w sposób jawny, a każda zmiana stanu gry (np. trafienie w statek) jest publicznie dostępna. Zwycięzca gry nie tylko otrzymuje pulę, ale ma również pełną możliwość udowodnienia swojej wygranej poprzez weryfikację zapisanych na blockchainie transakcji. To eliminuje ryzyko oszustwa czy manipulacji. Ponad to, system rankingowy jest w pełni jawny i deterministyczny, więc gracze mają pewność, że pozycje użytkowników w rankingu są uczciwie zdobyte.

\subsection{Prowizje i monetyzacja}

Zastosowanie blockchaina umożliwia właścicielowi kontraktów pobieranie prowizji od wygranych. System ten jest bezpieczny i automatyczny, co eliminuje potrzebę jakiejkolwiek zewnętrznej kontroli nad procesem wypłat. Prowizja będzie ustalana w sposób przejrzysty, a użytkownicy mogą mieć pewność, że zasady są jasno określone i egzekwowane.

\section{Zalety projektu}

Chainship to projekt o ogromnym potencjale, który przyciągnie zarówno graczy, jak i inwestorów.

\subsection{Innowacyjność}

Połączenie klasycznej gry w statki z technologią blockchain sprawia, że projekt wyróżnia się na tle tradycyjnych gier online. To innowacyjne podejście pozwala graczom cieszyć się rozrywką, mając jednocześnie pewność, że ich dane są przechowywane w sposób bezpieczny i transparentny.

\subsection{Rywalizacja}

Ranking graczy i rywalizacja o pulę pieniężną tworzy elementy gier hazardowych, ale w bezpiecznej, transparentnej i zdecentralizowanej formie. To z pewnością przyciągnie osoby szukające nowych wyzwań oraz tych, którzy lubią rywalizować w grach online.

\subsection{Potencjał na rynku blockchain}

Z uwagi na rosnące zainteresowanie grami na blockchainie i decentralizacją, projekt ma duży potencjał na rynku. Gry oparte na blockchainie są coraz bardziej popularne, a ten projekt może przyciągnąć zarówno entuzjastów blockchaina, jak i miłośników klasycznych gier komputerowych.

\subsection{Zautomatyzowana monetyzacja}

Możliwość pobierania prowizji od wygranych w połączeniu z automatycznym procesem wypłat stanowi solidny fundament monetyzacji. Gra będzie mogła generować stałe przychody dla właściciela kontraktów, co zapewnia długoterminową opłacalność projektu.

\subsection{Możliwość rozwoju}

Projekt może być rozwijany o dodatkowe funkcjonalności, takie jak nowe rodzaje gier, bazując na istniejącym rankingu graczy. Możliwość dodania nowych opcji rozgrywki sprawia, że projekt ma ogromny potencjał do rozwoju i dostosowania do potrzeb użytkowników.

\section{Podsumowanie}

Chainship to przełomowa aplikacja, która łączy klasyczną rozrywkę z rewolucyjną technologią blockchain. To projekt, który ma potencjał zmienić sposób, w jaki myślimy o grach online, oferując bezpieczeństwo, transparentność i uczciwość, które są niemożliwe do osiągnięcia w tradycyjnych grach. Chainship to nie tylko gra – to inwestycja w przyszłość gamingu.

\end{document}
